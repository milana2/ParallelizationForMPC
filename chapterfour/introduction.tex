\section{Introduction}
\label{sec:intro}

%\ana{Lindsey, look at the figures to format some of them and put the period at the end of the caption if it's missing.}

As the demand for cloud computation has increased so has the demand for secure solutions 
for cloud platforms. In the last decade cloud serves have become an almost \$100 billion industry. 
\cite{Synergy2019}. This high consumer demand, as well as the need for high security within these
frameworks has fueled research into varying cryptographic concepts like Multiparty Computation and 
Homomorphic Encryption. 

Researchers have proposed different encryption schemes to support 
a variety of operations, allowing for larger and larger subsets of
programs to be run this way. Again, Fully Homomorphic Encryption (FHE) \cite{Gentry:2010},\cite{Gentry:20102}  
can perform arbitrary operations on encrypted data, but in its current state FHE
is still prohibitively expensive \cite{Gentry:2011}. An alternative to FHE 
is Partially Homomorphic Encryption (PHE), which scales substantially better, but cannot 
perform arbitrary operations, and requires the conversions. %that were discussed in \chapref{chapter:chapterthree}.

As we discussed earlier, Multiparty Computation (MPC) describe groups of protocols that can be used in secure collaborative
computation. MPC benefits from having high security stemming from extensive theoretical guarantees. 
%without having to resort to expensive
%solutions (like FHE) or solutions with lower security (PHE conversion). 

Within MPC certain optimizations can be made in the evaluation of loops bodies. Due to the transformation of problems into an 
intermediate representation, similar operations within an MPC problem can be amortized. %, or performed in parallel, 
%with little drawbacks as long as certain assumptions hold. 
We present a technique for automatically locating opportunities for amortization to improve performance in compiled MPC programs. 

The rests of this chapter is organized as follows: \secref{sec:mpcoverview}, provides a brief over view of our work. 
\secref{sec:language} shows our language specification.
\secref{sec:parallelization} briefly outlines outlines dependenceies, and describes HPC Parallelizion as well as 
MPC Amortization in \secref{sec:hpcparallelization} and \secref{sec:mpcamortization}. 
\secref{sec:loopbodyanalysis} and \secref{sec:analysis} show our analysis. \secref{sec:loopscheduling} 
presents our scheduling algorithms. \secref{sec:results} presents our empirical results. 
% Finally, \secref{sec:conclusions} concludes. 
%\ana{Broken links again here. I think this needs to be rewritten as we moved stuff into Chapter 2.} \lindsey{fixed!}